% Example slides demonstrating all special environments defined in the template
% This file can be included in slides.tex using % Example slides demonstrating all special environments defined in the template
% This file can be included in slides.tex using % Example slides demonstrating all special environments defined in the template
% This file can be included in slides.tex using % Example slides demonstrating all special environments defined in the template
% This file can be included in slides.tex using \input{example_slides.tex}

\section{Special Environments in Slides}

\begin{frame}{Mathematical Environments}
    This slide demonstrates mathematical theorem-like environments available in Beamer.
    
    \begin{theorem}[Sample Theorem]
        In Beamer, theorems are displayed with colored borders and backgrounds.
    \end{theorem}
    
    \begin{lemma}[Sample Lemma]
        Lemmas use a different color scheme from theorems.
    \end{lemma}
    
    \begin{corollary}[Sample Corollary]
        Corollaries follow from theorems and lemmas.
    \end{corollary}
\end{frame}

\begin{frame}{Definitions and Examples}
    \begin{definition}[Important Concept]
        Definitions are used to introduce key mathematical or technical concepts.
        In Beamer, definitions have distinct styling to differentiate them from theorems.
    \end{definition}
    
    \begin{example}
        Examples help illustrate how definitions and theorems are applied in practice.
    \end{example}
\end{frame}

\begin{frame}{Propositions and Claims}
    \begin{prop}[Sample Proposition]
        Propositions are statements that can be proven or disproven.
    \end{prop}
    
    \begin{claim}[Sample Claim]
        Claims are assertions that require verification or proof.
    \end{claim}
\end{frame}

\begin{frame}{Proofs and Explanations}
    \begin{proof}[Sample Proof]
        This is a proof environment. In Beamer, proofs maintain the same QED symbol
        as defined in the preamble (smiley face by default).
    \end{proof}
    
    \begin{explanation}
        Explanations provide detailed reasoning or clarification of concepts.
    \end{explanation}
\end{frame}

\begin{frame}{Questions and Solutions}
    \begin{question}
        How do special environments work in Beamer slides?
    \end{question}
    
    \begin{solution}
        Special environments in Beamer use the same styling as in reports,
        with colored boxes and appropriate formatting for presentation.
    \end{solution}
\end{frame}

\begin{frame}{Notes and Custom Commands}
    \begin{notebox}
        Notes can be used to add side comments or additional information
        that doesn't fit in the main flow of the presentation.
        (Note: Use \texttt{notebox} instead of \texttt{note} since Beamer reserves \texttt{note} for speaker notes)
    \end{notebox}
    
    Custom commands also work in slides:
    \begin{itemize}
        \item \cmd{commandname} - formats commands in red
        \item \env{envname} - formats environments in blue
        \item \ulbf{bold underline} - creates bold underlined text
        \item \todo{reminder} - creates red italic todo notes
    \end{itemize}
\end{frame}

\begin{frame}{Mathematical Notation}
    The template includes custom mathematical commands that work in Beamer:
    
    \begin{itemize}
        \item KL divergence: $\infdiv{P}{Q}$ where $P$ and $Q$ are probability distributions
        \item Norm: $\norm{\mathbf{x}}$ for vector norms
        \item Expectation: $\E[X]$ for expected value
        \item Operators: $\argmax_{x} f(x)$, $\argmin_{x} f(x)$, $\vperp$
    \end{itemize}
    
    \vspace{0.5cm}
    Example equation:
    \begin{equation}
        \hat{\bm{s}}(\psi) = \bm{Z}^\text{post}_\psi - \bm{Z}^\text{pre}
    \end{equation}
\end{frame}

\begin{frame}[fragile]{Code Listings in Slides}
    The template supports code listings with syntax highlighting.
    Note: frames with code listings need the \texttt{[fragile]} option.
    
    \begin{lstlisting}[language=Python,basicstyle=\tiny\ttfamily]
def example_function(x, y):
    """Example Python function"""
    result = x + y
    return result
    \end{lstlisting}
    
    \begin{lstlisting}[language=R,basicstyle=\tiny\ttfamily]
# Example R code
x <- c(1, 2, 3, 4, 5)
mean(x)
    \end{lstlisting}
\end{frame}

\begin{frame}[fragile]{Algorithms in Slides}
    The template supports algorithm environments:
    
    \begin{algorithm}[H]
    \caption{Sample Algorithm}
    \begin{algorithmic}[1]
        \Initialize{Set initial parameters}
        \For{each iteration}
            \State Compute value
            \If{condition is met}
                \State Return result
            \EndIf
        \EndFor
    \end{algorithmic}
    \end{algorithm}
\end{frame}

\begin{frame}{Special Symbols}
    The template provides checkmark and cross symbols:
    \begin{itemize}
        \item \cmark{} - checkmark symbol (correct)
        \item \xmark{} - cross symbol (incorrect)
    \end{itemize}
    
    These are useful for indicating correct/incorrect answers or completed tasks.
\end{frame}

\begin{frame}{Summary}
    All special environments defined in the template work in Beamer slides:
    \begin{itemize}
        \item Mathematical: theorem, lemma, corollary, proposition, claim
        \item Pedagogical: definition, example, exercise, question, solution
        \item Proof-related: proof, explanation
        \item Informational: note
        \item Code: listings, algorithms
    \end{itemize}
    
    These environments maintain consistent styling across all document types.
\end{frame}

\begin{frame}{Generic Reusable Blocks}
    Use generic blocks with custom titles and colors for highlighting:
    
    \begin{purplevertblock}{Important Note}
    Vertical line style (like Exercise/Proof) with purple color.
    \end{purplevertblock}
    
    \begin{orangevertblock}{Custom Highlight}
    Works great for emphasizing important points in presentations.
    \end{orangevertblock}
\end{frame}

\begin{frame}{More Generic Blocks}
    \begin{goldtitleblock}{Key Concept}
    Title box style (like Definition) - perfect for key concepts.
    \end{goldtitleblock}
    
    \begin{goldborderblock}{Warning}
    Full border style (like Example) - useful for warnings.
    \end{goldborderblock}
    
    Use \texttt{vertlineblock}, \texttt{titleboxblock}, or \texttt{fullborderblock} with any color!
\end{frame}

\begin{frame}{Generic Blocks with Lists}
    \begin{greenvertblock}{Features}
    Generic blocks work with lists:
    \begin{itemize}
        \item Item 1
        \item Item 2
        \item Item 3
    \end{itemize}
    \end{greenvertblock}
    
    \begin{highlightblock}[myblue]{Reminder}
    Or use \texttt{highlightblock} for flexible highlighting with any color.
    \end{highlightblock}
\end{frame}



\section{Special Environments in Slides}

\begin{frame}{Mathematical Environments}
    This slide demonstrates mathematical theorem-like environments available in Beamer.
    
    \begin{theorem}[Sample Theorem]
        In Beamer, theorems are displayed with colored borders and backgrounds.
    \end{theorem}
    
    \begin{lemma}[Sample Lemma]
        Lemmas use a different color scheme from theorems.
    \end{lemma}
    
    \begin{corollary}[Sample Corollary]
        Corollaries follow from theorems and lemmas.
    \end{corollary}
\end{frame}

\begin{frame}{Definitions and Examples}
    \begin{definition}[Important Concept]
        Definitions are used to introduce key mathematical or technical concepts.
        In Beamer, definitions have distinct styling to differentiate them from theorems.
    \end{definition}
    
    \begin{example}
        Examples help illustrate how definitions and theorems are applied in practice.
    \end{example}
\end{frame}

\begin{frame}{Propositions and Claims}
    \begin{prop}[Sample Proposition]
        Propositions are statements that can be proven or disproven.
    \end{prop}
    
    \begin{claim}[Sample Claim]
        Claims are assertions that require verification or proof.
    \end{claim}
\end{frame}

\begin{frame}{Proofs and Explanations}
    \begin{proof}[Sample Proof]
        This is a proof environment. In Beamer, proofs maintain the same QED symbol
        as defined in the preamble (smiley face by default).
    \end{proof}
    
    \begin{explanation}
        Explanations provide detailed reasoning or clarification of concepts.
    \end{explanation}
\end{frame}

\begin{frame}{Questions and Solutions}
    \begin{question}
        How do special environments work in Beamer slides?
    \end{question}
    
    \begin{solution}
        Special environments in Beamer use the same styling as in reports,
        with colored boxes and appropriate formatting for presentation.
    \end{solution}
\end{frame}

\begin{frame}{Notes and Custom Commands}
    \begin{notebox}
        Notes can be used to add side comments or additional information
        that doesn't fit in the main flow of the presentation.
        (Note: Use \texttt{notebox} instead of \texttt{note} since Beamer reserves \texttt{note} for speaker notes)
    \end{notebox}
    
    Custom commands also work in slides:
    \begin{itemize}
        \item \cmd{commandname} - formats commands in red
        \item \env{envname} - formats environments in blue
        \item \ulbf{bold underline} - creates bold underlined text
        \item \todo{reminder} - creates red italic todo notes
    \end{itemize}
\end{frame}

\begin{frame}{Mathematical Notation}
    The template includes custom mathematical commands that work in Beamer:
    
    \begin{itemize}
        \item KL divergence: $\infdiv{P}{Q}$ where $P$ and $Q$ are probability distributions
        \item Norm: $\norm{\mathbf{x}}$ for vector norms
        \item Expectation: $\E[X]$ for expected value
        \item Operators: $\argmax_{x} f(x)$, $\argmin_{x} f(x)$, $\vperp$
    \end{itemize}
    
    \vspace{0.5cm}
    Example equation:
    \begin{equation}
        \hat{\bm{s}}(\psi) = \bm{Z}^\text{post}_\psi - \bm{Z}^\text{pre}
    \end{equation}
\end{frame}

\begin{frame}[fragile]{Code Listings in Slides}
    The template supports code listings with syntax highlighting.
    Note: frames with code listings need the \texttt{[fragile]} option.
    
    \begin{lstlisting}[language=Python,basicstyle=\tiny\ttfamily]
def example_function(x, y):
    """Example Python function"""
    result = x + y
    return result
    \end{lstlisting}
    
    \begin{lstlisting}[language=R,basicstyle=\tiny\ttfamily]
# Example R code
x <- c(1, 2, 3, 4, 5)
mean(x)
    \end{lstlisting}
\end{frame}

\begin{frame}[fragile]{Algorithms in Slides}
    The template supports algorithm environments:
    
    \begin{algorithm}[H]
    \caption{Sample Algorithm}
    \begin{algorithmic}[1]
        \Initialize{Set initial parameters}
        \For{each iteration}
            \State Compute value
            \If{condition is met}
                \State Return result
            \EndIf
        \EndFor
    \end{algorithmic}
    \end{algorithm}
\end{frame}

\begin{frame}{Special Symbols}
    The template provides checkmark and cross symbols:
    \begin{itemize}
        \item \cmark{} - checkmark symbol (correct)
        \item \xmark{} - cross symbol (incorrect)
    \end{itemize}
    
    These are useful for indicating correct/incorrect answers or completed tasks.
\end{frame}

\begin{frame}{Summary}
    All special environments defined in the template work in Beamer slides:
    \begin{itemize}
        \item Mathematical: theorem, lemma, corollary, proposition, claim
        \item Pedagogical: definition, example, exercise, question, solution
        \item Proof-related: proof, explanation
        \item Informational: note
        \item Code: listings, algorithms
    \end{itemize}
    
    These environments maintain consistent styling across all document types.
\end{frame}

\begin{frame}{Generic Reusable Blocks}
    Use generic blocks with custom titles and colors for highlighting:
    
    \begin{purplevertblock}{Important Note}
    Vertical line style (like Exercise/Proof) with purple color.
    \end{purplevertblock}
    
    \begin{orangevertblock}{Custom Highlight}
    Works great for emphasizing important points in presentations.
    \end{orangevertblock}
\end{frame}

\begin{frame}{More Generic Blocks}
    \begin{goldtitleblock}{Key Concept}
    Title box style (like Definition) - perfect for key concepts.
    \end{goldtitleblock}
    
    \begin{goldborderblock}{Warning}
    Full border style (like Example) - useful for warnings.
    \end{goldborderblock}
    
    Use \texttt{vertlineblock}, \texttt{titleboxblock}, or \texttt{fullborderblock} with any color!
\end{frame}

\begin{frame}{Generic Blocks with Lists}
    \begin{greenvertblock}{Features}
    Generic blocks work with lists:
    \begin{itemize}
        \item Item 1
        \item Item 2
        \item Item 3
    \end{itemize}
    \end{greenvertblock}
    
    \begin{highlightblock}[myblue]{Reminder}
    Or use \texttt{highlightblock} for flexible highlighting with any color.
    \end{highlightblock}
\end{frame}



\section{Special Environments in Slides}

\begin{frame}{Mathematical Environments}
    This slide demonstrates mathematical theorem-like environments available in Beamer.
    
    \begin{theorem}[Sample Theorem]
        In Beamer, theorems are displayed with colored borders and backgrounds.
    \end{theorem}
    
    \begin{lemma}[Sample Lemma]
        Lemmas use a different color scheme from theorems.
    \end{lemma}
    
    \begin{corollary}[Sample Corollary]
        Corollaries follow from theorems and lemmas.
    \end{corollary}
\end{frame}

\begin{frame}{Definitions and Examples}
    \begin{definition}[Important Concept]
        Definitions are used to introduce key mathematical or technical concepts.
        In Beamer, definitions have distinct styling to differentiate them from theorems.
    \end{definition}
    
    \begin{example}
        Examples help illustrate how definitions and theorems are applied in practice.
    \end{example}
\end{frame}

\begin{frame}{Propositions and Claims}
    \begin{prop}[Sample Proposition]
        Propositions are statements that can be proven or disproven.
    \end{prop}
    
    \begin{claim}[Sample Claim]
        Claims are assertions that require verification or proof.
    \end{claim}
\end{frame}

\begin{frame}{Proofs and Explanations}
    \begin{proof}[Sample Proof]
        This is a proof environment. In Beamer, proofs maintain the same QED symbol
        as defined in the preamble (smiley face by default).
    \end{proof}
    
    \begin{explanation}
        Explanations provide detailed reasoning or clarification of concepts.
    \end{explanation}
\end{frame}

\begin{frame}{Questions and Solutions}
    \begin{question}
        How do special environments work in Beamer slides?
    \end{question}
    
    \begin{solution}
        Special environments in Beamer use the same styling as in reports,
        with colored boxes and appropriate formatting for presentation.
    \end{solution}
\end{frame}

\begin{frame}{Notes and Custom Commands}
    \begin{notebox}
        Notes can be used to add side comments or additional information
        that doesn't fit in the main flow of the presentation.
        (Note: Use \texttt{notebox} instead of \texttt{note} since Beamer reserves \texttt{note} for speaker notes)
    \end{notebox}
    
    Custom commands also work in slides:
    \begin{itemize}
        \item \cmd{commandname} - formats commands in red
        \item \env{envname} - formats environments in blue
        \item \ulbf{bold underline} - creates bold underlined text
        \item \todo{reminder} - creates red italic todo notes
    \end{itemize}
\end{frame}

\begin{frame}{Mathematical Notation}
    The template includes custom mathematical commands that work in Beamer:
    
    \begin{itemize}
        \item KL divergence: $\infdiv{P}{Q}$ where $P$ and $Q$ are probability distributions
        \item Norm: $\norm{\mathbf{x}}$ for vector norms
        \item Expectation: $\E[X]$ for expected value
        \item Operators: $\argmax_{x} f(x)$, $\argmin_{x} f(x)$, $\vperp$
    \end{itemize}
    
    \vspace{0.5cm}
    Example equation:
    \begin{equation}
        \hat{\bm{s}}(\psi) = \bm{Z}^\text{post}_\psi - \bm{Z}^\text{pre}
    \end{equation}
\end{frame}

\begin{frame}[fragile]{Code Listings in Slides}
    The template supports code listings with syntax highlighting.
    Note: frames with code listings need the \texttt{[fragile]} option.
    
    \begin{lstlisting}[language=Python,basicstyle=\tiny\ttfamily]
def example_function(x, y):
    """Example Python function"""
    result = x + y
    return result
    \end{lstlisting}
    
    \begin{lstlisting}[language=R,basicstyle=\tiny\ttfamily]
# Example R code
x <- c(1, 2, 3, 4, 5)
mean(x)
    \end{lstlisting}
\end{frame}

\begin{frame}[fragile]{Algorithms in Slides}
    The template supports algorithm environments:
    
    \begin{algorithm}[H]
    \caption{Sample Algorithm}
    \begin{algorithmic}[1]
        \Initialize{Set initial parameters}
        \For{each iteration}
            \State Compute value
            \If{condition is met}
                \State Return result
            \EndIf
        \EndFor
    \end{algorithmic}
    \end{algorithm}
\end{frame}

\begin{frame}{Special Symbols}
    The template provides checkmark and cross symbols:
    \begin{itemize}
        \item \cmark{} - checkmark symbol (correct)
        \item \xmark{} - cross symbol (incorrect)
    \end{itemize}
    
    These are useful for indicating correct/incorrect answers or completed tasks.
\end{frame}

\begin{frame}{Summary}
    All special environments defined in the template work in Beamer slides:
    \begin{itemize}
        \item Mathematical: theorem, lemma, corollary, proposition, claim
        \item Pedagogical: definition, example, exercise, question, solution
        \item Proof-related: proof, explanation
        \item Informational: note
        \item Code: listings, algorithms
    \end{itemize}
    
    These environments maintain consistent styling across all document types.
\end{frame}

\begin{frame}{Generic Reusable Blocks}
    Use generic blocks with custom titles and colors for highlighting:
    
    \begin{purplevertblock}{Important Note}
    Vertical line style (like Exercise/Proof) with purple color.
    \end{purplevertblock}
    
    \begin{orangevertblock}{Custom Highlight}
    Works great for emphasizing important points in presentations.
    \end{orangevertblock}
\end{frame}

\begin{frame}{More Generic Blocks}
    \begin{goldtitleblock}{Key Concept}
    Title box style (like Definition) - perfect for key concepts.
    \end{goldtitleblock}
    
    \begin{goldborderblock}{Warning}
    Full border style (like Example) - useful for warnings.
    \end{goldborderblock}
    
    Use \texttt{vertlineblock}, \texttt{titleboxblock}, or \texttt{fullborderblock} with any color!
\end{frame}

\begin{frame}{Generic Blocks with Lists}
    \begin{greenvertblock}{Features}
    Generic blocks work with lists:
    \begin{itemize}
        \item Item 1
        \item Item 2
        \item Item 3
    \end{itemize}
    \end{greenvertblock}
    
    \begin{highlightblock}[myblue]{Reminder}
    Or use \texttt{highlightblock} for flexible highlighting with any color.
    \end{highlightblock}
\end{frame}



\section{Special Environments in Slides}

\begin{frame}{Mathematical Environments}
    This slide demonstrates mathematical theorem-like environments available in Beamer.
    
    \begin{theorem}[Sample Theorem]
        In Beamer, theorems are displayed with colored borders and backgrounds.
    \end{theorem}
    
    \begin{lemma}[Sample Lemma]
        Lemmas use a different color scheme from theorems.
    \end{lemma}
    
    \begin{corollary}[Sample Corollary]
        Corollaries follow from theorems and lemmas.
    \end{corollary}
\end{frame}

\begin{frame}{Definitions and Examples}
    \begin{definition}[Important Concept]
        Definitions are used to introduce key mathematical or technical concepts.
        In Beamer, definitions have distinct styling to differentiate them from theorems.
    \end{definition}
    
    \begin{example}
        Examples help illustrate how definitions and theorems are applied in practice.
    \end{example}
\end{frame}

\begin{frame}{Propositions and Claims}
    \begin{prop}[Sample Proposition]
        Propositions are statements that can be proven or disproven.
    \end{prop}
    
    \begin{claim}[Sample Claim]
        Claims are assertions that require verification or proof.
    \end{claim}
\end{frame}

\begin{frame}{Proofs and Explanations}
    \begin{proof}[Sample Proof]
        This is a proof environment. In Beamer, proofs maintain the same QED symbol
        as defined in the preamble (smiley face by default).
    \end{proof}
    
    \begin{explanation}
        Explanations provide detailed reasoning or clarification of concepts.
    \end{explanation}
\end{frame}

\begin{frame}{Questions and Solutions}
    \begin{question}
        How do special environments work in Beamer slides?
    \end{question}
    
    \begin{solution}
        Special environments in Beamer use the same styling as in reports,
        with colored boxes and appropriate formatting for presentation.
    \end{solution}
\end{frame}

\begin{frame}{Notes and Custom Commands}
    \begin{notebox}
        Notes can be used to add side comments or additional information
        that doesn't fit in the main flow of the presentation.
        (Note: Use \texttt{notebox} instead of \texttt{note} since Beamer reserves \texttt{note} for speaker notes)
    \end{notebox}
    
    Custom commands also work in slides:
    \begin{itemize}
        \item \cmd{commandname} - formats commands in red
        \item \env{envname} - formats environments in blue
        \item \ulbf{bold underline} - creates bold underlined text
        \item \todo{reminder} - creates red italic todo notes
    \end{itemize}
\end{frame}

\begin{frame}{Mathematical Notation}
    The template includes custom mathematical commands that work in Beamer:
    
    \begin{itemize}
        \item KL divergence: $\infdiv{P}{Q}$ where $P$ and $Q$ are probability distributions
        \item Norm: $\norm{\mathbf{x}}$ for vector norms
        \item Expectation: $\E[X]$ for expected value
        \item Operators: $\argmax_{x} f(x)$, $\argmin_{x} f(x)$, $\vperp$
    \end{itemize}
    
    \vspace{0.5cm}
    Example equation:
    \begin{equation}
        \hat{\bm{s}}(\psi) = \bm{Z}^\text{post}_\psi - \bm{Z}^\text{pre}
    \end{equation}
\end{frame}

\begin{frame}[fragile]{Code Listings in Slides}
    The template supports code listings with syntax highlighting.
    Note: frames with code listings need the \texttt{[fragile]} option.
    
    \begin{lstlisting}[language=Python,basicstyle=\tiny\ttfamily]
def example_function(x, y):
    """Example Python function"""
    result = x + y
    return result
    \end{lstlisting}
    
    \begin{lstlisting}[language=R,basicstyle=\tiny\ttfamily]
# Example R code
x <- c(1, 2, 3, 4, 5)
mean(x)
    \end{lstlisting}
\end{frame}

\begin{frame}[fragile]{Algorithms in Slides}
    The template supports algorithm environments:
    
    \begin{algorithm}[H]
    \caption{Sample Algorithm}
    \begin{algorithmic}[1]
        \Initialize{Set initial parameters}
        \For{each iteration}
            \State Compute value
            \If{condition is met}
                \State Return result
            \EndIf
        \EndFor
    \end{algorithmic}
    \end{algorithm}
\end{frame}

\begin{frame}{Special Symbols}
    The template provides checkmark and cross symbols:
    \begin{itemize}
        \item \cmark{} - checkmark symbol (correct)
        \item \xmark{} - cross symbol (incorrect)
    \end{itemize}
    
    These are useful for indicating correct/incorrect answers or completed tasks.
\end{frame}

\begin{frame}{Summary}
    All special environments defined in the template work in Beamer slides:
    \begin{itemize}
        \item Mathematical: theorem, lemma, corollary, proposition, claim
        \item Pedagogical: definition, example, exercise, question, solution
        \item Proof-related: proof, explanation
        \item Informational: note
        \item Code: listings, algorithms
    \end{itemize}
    
    These environments maintain consistent styling across all document types.
\end{frame}

\begin{frame}{Generic Reusable Blocks}
    Use generic blocks with custom titles and colors for highlighting:
    
    \begin{purplevertblock}{Important Note}
    Vertical line style (like Exercise/Proof) with purple color.
    \end{purplevertblock}
    
    \begin{orangevertblock}{Custom Highlight}
    Works great for emphasizing important points in presentations.
    \end{orangevertblock}
\end{frame}

\begin{frame}{More Generic Blocks}
    \begin{goldtitleblock}{Key Concept}
    Title box style (like Definition) - perfect for key concepts.
    \end{goldtitleblock}
    
    \begin{goldborderblock}{Warning}
    Full border style (like Example) - useful for warnings.
    \end{goldborderblock}
    
    Use \texttt{vertlineblock}, \texttt{titleboxblock}, or \texttt{fullborderblock} with any color!
\end{frame}

\begin{frame}{Generic Blocks with Lists}
    \begin{greenvertblock}{Features}
    Generic blocks work with lists:
    \begin{itemize}
        \item Item 1
        \item Item 2
        \item Item 3
    \end{itemize}
    \end{greenvertblock}
    
    \begin{highlightblock}[myblue]{Reminder}
    Or use \texttt{highlightblock} for flexible highlighting with any color.
    \end{highlightblock}
\end{frame}

