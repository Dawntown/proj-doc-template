% Basic packages and settings for all outputs
% This file contains common packages used across manuscript, report, and slides

% Essential packages
\usepackage{amsmath}        % Advanced math formatting
\usepackage{amssymb}        % Advanced math symbols
\usepackage{amsthm}         % Theorem environments
\usepackage{graphicx}       % Graphics support
\usepackage{hyperref}       % Hyperlinks
\usepackage{booktabs}       % Professional tables
\usepackage{float}          % Better float control
\usepackage{xcolor}         % Color support
\usepackage{microtype}      % Better typography
\usepackage{bm}             % Bold math
\usepackage{subcaption}     % Subfigure environment
\usepackage{natbib}         % Advanced citation management
\usepackage{caption}        % Caption formatting
\usepackage{tcolorbox}
\usepackage{xparse}         % Enhanced command definitions
\usepackage[ruled,vlined,linesnumbered]{algorithm2e} % Algorithm environment

% Hyperref settings
% Note: research-collection does not use colorlinks, so links appear as boxes instead of colored text
% Uncomment the following to use colored links:
% \hypersetup{
%     colorlinks,
%     linkcolor=blue,
%     citecolor=black,
%     urlcolor=blue!80!black
% }

% Caption settings
\captionsetup[figure]{labelfont=bf, labelsep=period, name={Figure}}
\captionsetup[table]{labelfont=bf, labelsep=period, name={Table}}

% Cross-reference settings
\usepackage{cleveref}       % Smart cross-references
\Crefname{figure}{Figure}{Figures}
\crefname{figure}{Figure}{Figures}
\Crefname{table}{Table}{Tables}
\crefname{table}{Table}{Tables}
\Crefname{equation}{Equation}{Equations}
\crefname{equation}{Equation}{Equations}

%%%%%%%%%%%%%%%%%%%%%%%%%%%%%%%%%%%%%%%%%%%%%%%%%%%%%%%%%%%%%%%%%%%%%%%%%%%%%%%
%  Simple and Useful Symbol Definitions                                        %
%%%%%%%%%%%%%%%%%%%%%%%%%%%%%%%%%%%%%%%%%%%%%%%%%%%%%%%%%%%%%%%%%%%%%%%%%%%%%%%

% Symbol aliases (common mathematical notation shortcuts)
\let\implies\Rightarrow
\let\impliedby\Leftarrow
\let\iff\Leftrightarrow
\let\epsilon\varepsilon

% Math operators
\DeclareMathOperator*{\argmax}{\arg\!max}
\DeclareMathOperator*{\argmin}{\arg\!min}
\DeclareMathOperator*{\vperp}{\text{\rotatebox{90}{$\models$}}}

% Paired delimiters and norms (requires mathtools)
\RequirePackage{mathtools}
\DeclarePairedDelimiterX{\infdivx}[2]{(}{)}{%
  #1\;\delimsize\|\;#2%
}
\DeclarePairedDelimiter{\norm}{\lVert}{\rVert}

% Common math commands
\newcommand{\E}{\mathbb{E}}
\newcommand{\infdiv}{D_{\mathbb{KL}}\infdivx}
\newcommand{\fisherdiv}{D_{\mathbb{F}}\infdivx}

% Simple utility commands
\newcommand*\widefbox[1]{\fbox{\hspace{2em}#1\hspace{2em}}}



% Custom counter for figure placeholder items
\newcounter{figureplaceholderitem}
\renewcommand{\thefigureplaceholderitem}{\Alph{figureplaceholderitem}}

% Custom item command that takes optional name
\newcommand{\figureplaceholderitem}[1][]{%
    \stepcounter{figureplaceholderitem}%
    \par\noindent\textbf{\thefigureplaceholderitem.}%
    \ifx\relax#1\relax%
    \else%
        \textbf{ #1}%
    \fi%
    \space
}

% Helper command for item in figureplaceholder environment
% Use xparse for more robust optional argument handling
\makeatletter
% Use xparse for better compatibility with enumitem
\NewDocumentCommand{\@figureplaceholderitemcmd}{o}{%
    \IfValueTF{#1}{%
        \figureplaceholderitem[#1]%
    }{%
        \figureplaceholderitem[]%
    }%
}
\makeatother

% Custom environment for figure placeholders
\makeatletter
\newtcolorbox{figureplaceholder}{
    fontupper=\footnotesize,
    boxrule=0.5pt,
    colback=white,
    colframe=black,
    left=5pt,
    right=5pt,
    top=5pt,
    bottom=5pt,
    before upper={%
        \linespread{1}\selectfont%
        \setcounter{figureplaceholderitem}{0}%
        \let\olditem\item%
        \let\item\@figureplaceholderitemcmd%
    },
    after upper={%
        \let\item\olditem%
        \setcounter{figureplaceholderitem}{0}%
    }
}
\makeatother